%% LyX 2.3.7 created this file.  For more info, see http://www.lyx.org/.
%% Do not edit unless you really know what you are doing.
\documentclass[twocolumn,english,notitlepage]{revtex4-1}
\usepackage[T1]{fontenc}
\usepackage[latin9]{inputenc}
\setcounter{secnumdepth}{3}
\usepackage{amsmath}
\usepackage{amssymb}
\usepackage{babel}
\usepackage{listings}
\renewcommand{\lstlistingname}{Listing}

\begin{document}
\title{Equilibrium model B}
\maketitle

\section{Conserved Dynamics}

Let us define $\phi(\mathbf{r},t)$ to be the rescaled density. The
coarse-grained Hamiltonian can be written as:
\begin{equation}
\mathcal{H}[\phi]=\int_{V}d\mathbf{r}\left\{ \frac{a}{2}\phi^{2}+\frac{b}{4}\phi^{4}+\frac{\kappa}{2}|\nabla\phi|^{2}\right\} ,\label{eq:H}
\end{equation}
where $b,\kappa>0$ (otherwise the energy is not bounded from below).
$a$ can be positive or negative. The dynamics then follows the conservation
law:
\begin{align}
\frac{\partial\phi}{\partial t}+\nabla\cdot\mathbf{J} & =0\\
\mathbf{J} & =-\lambda\nabla\frac{\delta\mathcal{H}}{\delta\phi}+\boldsymbol{\Lambda},\label{eq:J}
\end{align}
where $\lambda>0$. Correspondingly, the global density $\phi_{0}=\frac{1}{V}\int\phi\,d\mathbf{r}$
is constant with time. $\boldsymbol{\Lambda}(\mathbf{r},t)$ in (\ref{eq:J})
is Gaussian white noise with zero mean and Dirac delta-correlation:
\begin{equation}
\left\langle \Lambda_{\alpha}(\mathbf{r},t)\Lambda_{\beta}(\mathbf{r}',t')\right\rangle =2\lambda T\delta_{\alpha\beta}\delta(\mathbf{r}-\mathbf{r}')\delta(t-t').
\end{equation}
The noise correlation above satisfies FDT, which guarantees that the
system will be in thermal equilibrium with a heat bath of temperature
$T$ at steady state $t\rightarrow\infty$.

The equilibrium state of the system depends on the values of $a$
and $\phi_{0}$:
\begin{itemize}
\item $a>0$ or $a<0$ and $|\phi_{0}|>\sqrt{\frac{-a}{b}}$: homogenous
state
\item $a<0$ and $|\phi_{0}|<\sqrt{\frac{-a}{b}}$: phase separated state.
\end{itemize}

\section{Steady state statistics}

In the steady state $t\rightarrow\infty$, and for a fixed value of
$a$, the probability of obtaining some configuration $\phi(\mathbf{r})$,
for any time $t$, is given by the Boltzmann distribution:
\begin{equation}
P_{s}[\phi(\mathbf{r})]=\frac{1}{\mathcal{Z}}e^{-\mathcal{H}[\phi(\mathbf{r})]/T},\label{eq:P-s}
\end{equation}
where $\mathcal{Z}$ is the partition function:
\begin{equation}
\mathcal{Z}=\int\mathcal{D}\phi\,e^{-\mathcal{H}[\phi]/T}.
\end{equation}
Note that the Hamiltonian $\mathcal{H}[\phi]$ is a fluctuating quantity
since $\phi$ is a fluctuating field. To get the thermodynamic energy,
we then have to average $\mathcal{H}[\phi]$ over the stationary distribution
$P_{s}[\phi]$:
\begin{equation}
\left\langle \mathcal{H}\right\rangle _{s}=\int\mathcal{D}\phi\,\mathcal{H}[\phi]P_{s}[\phi],
\end{equation}
where in the above $\left\langle \dots\right\rangle _{s}$ indicates
averaging over stationary distribution $P_{s}[\phi]$. Now the thermodynamic
entropy of the system $\mathcal{S}$ is defined to be:
\begin{align}
\mathcal{S} & =-\left\langle \ln P_{s}\right\rangle _{s}.
\end{align}
Substituting (\ref{eq:P-s}) to the above equation, we then derive
the \emph{total }free energy of the system:
\begin{equation}
\mathcal{F}=-T\ln\mathcal{Z}=\left\langle \mathcal{H}\right\rangle _{s}-T\mathcal{S}.\label{eq:F}
\end{equation}
Note that in some literature $\mathcal{H}$ is sometimes called the
coarse-grained free energy and $\mathcal{F}$ is the \emph{total}
free energy.

\begin{widetext}

\section{Gaussian approximation}

Let us consider the equilibrium homogenous state. In steady state,
we have an ensemble of different configurations $\phi(\mathbf{r})$'s
from different time steps. Let us now write $\phi(\mathbf{r})$ as:
\begin{equation}
\phi(\mathbf{r})=\underbrace{\phi_{0}}_{\text{mean field}}+\underbrace{\delta\phi(\mathbf{r})}_{\text{fluctuations around mean field}},
\end{equation}
where $\delta\phi(\mathbf{r})$ is assumed to be small. Substituting
the above into the Hamiltonian $\mathcal{H}[\phi]$, we get:
\begin{align}
\mathcal{H}[\phi] & =\int_{V}d\mathbf{r}\left\{ \frac{a}{2}(\phi_{0}+\delta\phi)^{2}+\frac{b}{4}(\phi_{0}+\delta\phi)^{4}+\frac{\kappa}{2}|\nabla\delta\phi|^{2}\right\} \\
 & \simeq\int_{V}d\mathbf{r}\left\{ \frac{a}{2}(\phi_{0}^{2}+2\phi_{0}\delta\phi+\delta\phi^{2})+\frac{b}{4}(\phi_{0}^{4}+4\phi_{0}^{3}\delta\phi+6\phi_{0}^{2}\delta\phi^{2})+\frac{\kappa}{2}|\nabla\delta\phi|^{2}\right\} ,
\end{align}
where we have ignored higher order terms $\sim\delta\phi^{3}$. Now
since $\phi$ is conserved, we must have $\int_{V}\delta\phi\,d\mathbf{r}=0$,
and thus:
\begin{equation}
\mathcal{H}[\delta\phi]\simeq\underbrace{V\left(\frac{a}{2}\phi_{0}^{2}+\frac{b}{4}\phi_{0}^{4}\right)}_{\mathcal{H}_{0}}+\int_{V}d\mathbf{r}\left\{ \left(\frac{a}{2}+\frac{3b\phi_{0}^{2}}{2}\right)\delta\phi^{2}+\frac{\kappa}{2}|\nabla\delta\phi|^{2}\right\} .
\end{equation}
The first term $\mathcal{H}_{0}=$ constant is the mean field energy.
Let us consider a $d$-dimenional box as our system. Now we can define
the Fourier transform of $\delta\phi(\mathbf{r})$:
\begin{align}
\delta\phi(\mathbf{r}) & =\frac{1}{\sqrt{V}}\sum_{\mathbf{q}}\delta\phi_{\mathbf{q}}e^{i\mathbf{q}\cdot\mathbf{r}}\\
\delta\phi_{\mathbf{q}} & =\frac{1}{\sqrt{V}}\int_{V}d\mathbf{r}\,\delta\phi(\mathbf{r})e^{-i\mathbf{q}\cdot\mathbf{r}},
\end{align}
where $V=L^{d}$ and 
\begin{equation}
q_{\alpha}=0,\pm\frac{2\pi}{L},\pm\frac{4\pi}{L},\pm\frac{6\pi}{L},\dots\quad\text{, where }\alpha=1,2,\dots,d.
\end{equation}
More succintly, we can also write:
\[
\mathbf{q}=\frac{2\pi}{L}\mathbf{n}\quad\text{, where }\mathbf{n}\in\mathbb{Z}^{d}.
\]
The Hamiltonian then becomes:
\begin{align}
\mathcal{H}\{\delta\phi_{\mathbf{q}}\} & =\mathcal{H}_{0}+\int_{V}d\mathbf{r}\left\{ \left(\frac{a}{2}+\frac{3b\phi_{0}^{2}}{2}\right)\frac{1}{V}\sum_{\mathbf{q},\mathbf{q}'}\delta\phi_{\mathbf{q}}\delta\phi_{\mathbf{q}'}e^{i(\mathbf{q}+\mathbf{q})\cdot\mathbf{r}}+\frac{\kappa}{2}\frac{1}{V}\sum_{\mathbf{q},\mathbf{q}'}(i\mathbf{q})\cdot(i\mathbf{q}')\delta\phi_{\mathbf{q}}\delta\phi_{\mathbf{q}'}e^{i(\mathbf{q}+\mathbf{q})\cdot\mathbf{r}}\right\} \\
 & =\mathcal{H}_{0}+\sum_{\mathbf{q},\mathbf{q}'}\left(\frac{a}{2}+\frac{3b\phi_{0}^{2}}{2}\right)\delta\phi_{\mathbf{q}}\delta\phi_{\mathbf{q}'}\underbrace{\frac{1}{V}\int_{V}d\mathbf{r}e^{i(\mathbf{q}+\mathbf{q}')\cdot\mathbf{r}}}_{\delta_{\mathbf{q},-\mathbf{q}'}}+\sum_{\mathbf{q},\mathbf{q}'}\frac{\kappa}{2}(i\mathbf{q})\cdot(i\mathbf{q}')\delta\phi_{\mathbf{q}}\delta\phi_{\mathbf{q}'}\underbrace{\frac{1}{V}\int_{V}d\mathbf{r}e^{i(\mathbf{q}+\mathbf{q}')\cdot\mathbf{r}}}_{\delta_{\mathbf{q},\mathbf{q}'}}\\
 & =\mathcal{H}_{0}+\frac{1}{2}\sum_{\mathbf{q}}\left(a+3b\phi_{0}^{2}+\kappa q^{2}\right)|\delta\phi_{\mathbf{q}}|^{2}
\end{align}

\end{widetext}

To simplify the notation, let us define:
\begin{equation}
G(\mathbf{q})=\frac{a+3b\phi_{0}^{2}+\kappa q^{2}}{T},
\end{equation}
so that the stationary probability distribution becomes:
\begin{align}
P_{s}\{\delta\phi_{\mathbf{q}}\} & =\frac{1}{\mathcal{Z}}e^{-\frac{1}{2}\sum_{\mathbf{q}}G(\mathbf{q})|\delta\phi_{\mathbf{q}}|^{2}}\\
\mathcal{Z} & =\left(\prod_{\mathbf{q}}\int d\delta\phi_{\mathbf{q}}\right)e^{-\frac{1}{2}\sum_{\mathbf{q}}G(\mathbf{q})|\delta\phi_{\mathbf{q}}|^{2}}
\end{align}
Note that since $\mathcal{H}_{0}=$ constant, we can absorb it inside
$\mathcal{Z}$. Now we can compute $\mathcal{Z}$:
\begin{align}
\mathcal{Z} & =\left(\prod_{\mathbf{q}}\int d\delta\phi_{\mathbf{q}}\right)e^{-\frac{1}{2}\sum_{\mathbf{q}}G(\mathbf{q})|\delta\phi_{\mathbf{q}}|^{2}}\\
 & =\prod_{\mathbf{q}}\left(\int d\delta\phi_{\mathbf{q}}\,e^{-\frac{1}{2}G(\mathbf{q})|\delta\phi_{\mathbf{q}}|^{2}}\right).
\end{align}
The integral inside the round bracket is a Gaussian integral over
the two random variables: $\text{Re}(\delta\phi_{\mathbf{q}})$ and
$\text{Im}(\delta\phi_{\mathbf{q}})$. However these two variables
are not independent since $\delta\phi_{\mathbf{q}}=\delta\phi_{-\mathbf{q}}^{*}$,
and effectively, this is just a one-dimensional Gaussian integral.
Thus,
\begin{equation}
\mathcal{Z}=\prod_{\mathbf{q}}\sqrt{\frac{2\pi}{G(\mathbf{q})}}.
\end{equation}
In particular, we can calculate the total free energy:
\begin{align}
\mathcal{F} & =-T\ln\mathcal{Z}\\
 & =-\frac{T}{2}\sum_{\mathbf{q}}\ln\left(\frac{2\pi}{G(\mathbf{q})}\right)\\
 & \simeq-T\frac{V}{(2\pi)^{d}}\int_{0}^{q_{\text{max}}}dq\,\Omega_{d}q^{d-1}\ln\left(\frac{2\pi}{G(\mathbf{q})}\right),
\end{align}
where $\Omega_{d}$ is the solid angle in $d$-dimension:
\begin{equation}
\Omega_{d}=\frac{2\pi^{d/2}}{\Gamma(d/2)},
\end{equation}
and $q_{\text{max}}$ is the cutoff wavevector. Typically $q_{\text{max}}\simeq\pi/\Delta x$,
where $\Delta x$ is the lattice size.

\section{Spatial correlation}

The spatial correlation function is defined to be:
\begin{equation}
C(\mathbf{r},\mathbf{r}')=\left\langle \delta\phi(\mathbf{r})\delta\phi(\mathbf{r}')\right\rangle _{s}.
\end{equation}
This measures the correlation of the density field at $\mathbf{r}$
and $\mathbf{r}'$. Substituting the definition of Fourier transform,
we get:
\begin{equation}
C(\mathbf{r},\mathbf{r}')=\frac{1}{V}\sum_{\mathbf{q},\mathbf{q}'}\left\langle \delta\phi_{\mathbf{q}}\delta\phi_{\mathbf{q}'}\right\rangle _{s}e^{i\mathbf{q}\cdot\mathbf{r}}e^{i\mathbf{q}'\cdot\mathbf{r}'}.
\end{equation}
However, since we have translational symmetry, we must $C(\mathbf{r},\mathbf{r}')$
only depends on $\mathbf{r}-\mathbf{r}'$, \emph{i.e.}, $C(\mathbf{r},\mathbf{r}')=C(\mathbf{r}-\mathbf{r}')$.
Thus, $\left\langle \delta\phi_{\mathbf{q}}\delta\phi_{\mathbf{q}'}\right\rangle _{s}$
must have the following form:
\begin{equation}
\left\langle \delta\phi_{\mathbf{q}}\delta\phi_{\mathbf{q}'}\right\rangle _{s}=\left\langle |\delta\phi_{\mathbf{q}}|^{2}\right\rangle _{s}\delta_{\mathbf{q},-\mathbf{q}'}
\end{equation}
so that
\begin{equation}
C(\mathbf{r}-\mathbf{r}')=\frac{1}{V}\sum_{\mathbf{q}}\underbrace{\left\langle |\delta\phi_{\mathbf{q}}|^{2}\right\rangle _{s}}_{S(\mathbf{q})}e^{i\mathbf{q}\cdot(\mathbf{r}-\mathbf{r}')}
\end{equation}
is a function of $\mathbf{r}-\mathbf{r}'$ only. $S(\mathbf{q})=\left\langle |\delta\phi_{\mathbf{q}}|^{2}\right\rangle _{s}$,
which is the Fourier transform of $C(\mathbf{r})$, is called the
structure factor. 

For Gaussian statistics,  the partition function can be written as:
\begin{equation}
\mathcal{Z}=\left(\prod_{\mathbf{q}}\int d\delta\phi_{\mathbf{q}}\right)e^{-\frac{1}{2}\sum_{\mathbf{q}}G(\mathbf{q})|\delta\phi_{\mathbf{q}}|^{2}}
\end{equation}
Now consider:
\begin{align}
\frac{1}{\mathcal{Z}}\frac{\partial\mathcal{Z}}{\partial G(\mathbf{q})} & =-\frac{1}{2}\left(\prod_{\mathbf{q}}\int d\delta\phi_{\mathbf{q}}\right)|\delta\phi_{\mathbf{q}}|^{2}\frac{1}{\mathcal{Z}}e^{-\frac{1}{2}\sum_{\mathbf{q}}G(\mathbf{q})|\delta\phi_{\mathbf{q}}|^{2}}\\
 & =-\frac{1}{2}\left(\prod_{\mathbf{q}}\int d\delta\phi_{\mathbf{q}}\right)|\delta\phi_{\mathbf{q}}|^{2}P_{s}\{\delta\phi_{\mathbf{q}}\}\\
 & =-\frac{1}{2}\left\langle |\delta\phi_{\mathbf{q}}|^{2}\right\rangle .
\end{align}
Thus we obtain the formula for the structure factor from the partition
function:
\begin{equation}
S(\mathbf{q})=\left\langle |\delta\phi_{\mathbf{q}}|^{2}\right\rangle _{s}=-2\frac{\partial\ln\mathcal{Z}}{\partial G(\mathbf{q})}.
\end{equation}
Using the expression for $\ln\mathcal{Z}$, we compute above, we can
find:
\begin{align}
S(\mathbf{q}) & =\frac{\partial}{\partial G(\mathbf{q})}\sum_{\mathbf{q}'}\ln\left(\frac{G(\mathbf{q}')}{2\pi}\right)\\
 & =\frac{1}{G(\mathbf{q})}\\
 & =\frac{T}{a+3b\phi_{0}^{2}+\kappa q^{2}}.
\end{align}
Let's consider $d=2$ as an example. The correlation function is then
just the Fourier transform of $S(\mathbf{q})$:
\begin{align}
C(\mathbf{r}) & =\frac{1}{L^{2}}\sum_{\mathbf{q}}\frac{T}{a+3b\phi_{0}^{2}+\kappa q^{2}}e^{i\mathbf{q}\cdot\mathbf{r}}\\
 & \simeq\frac{1}{(2\pi)^{2}}\int d\mathbf{q}\,\frac{T}{a+3b\phi_{0}^{2}+\kappa q^{2}}e^{i\mathbf{q}\cdot\mathbf{r}}.
\end{align}
Now we can change to polar coordinate, and without loss of generality,
we can assume $\mathbf{r}$ to be pointing along $x$-axis, or $\mathbf{r}=r\hat{\mathbf{x}}$:
\begin{align}
C(\mathbf{r}) & =\frac{1}{(2\pi)^{2}}\int_{0}^{q_{\text{max}}}dq\int_{0}^{2\pi}d\theta\,q\frac{T}{a+3b\phi_{0}^{2}+\kappa q^{2}}e^{iqr\cos\theta}\\
 & =\frac{1}{2\pi}\int_{0}^{q_{\text{max}}}dq\,\frac{Tq}{a+3b\phi_{0}^{2}+\kappa q^{2}}J_{0}(qr)\\
 & =\frac{T}{2\pi\kappa}\int_{0}^{q_{\text{max}}\xi}d(q\xi)\,\frac{q\xi}{1+(q\xi)^{2}}J_{0}\left(q\xi\frac{r}{\xi}\right)
\end{align}
where $J_{n}(x)$ is the Bessel function of first kind, and we have
introduced the correlation length:
\[
\xi=\sqrt{\frac{\kappa}{a+3b\phi_{0}^{2}}}.
\]
The correlation is infinite at the critical point $a=\phi_{0}=0$.
Now we are interested in the behaviour $r\gg\xi$ or $q\xi\ll1$,
we get:
\[
C(r)\simeq\frac{T}{2\pi\kappa}\int_{0}^{q_{\text{max}}\xi}d(q\xi)\,q\xi J_{0}\left(q\xi\frac{r}{\xi}\right)
\]


\section{Numerical simulations}

Let's consider $d=1$ system for now. The generalization to higher
dimension is straightforward. The equation we are solving is:
\begin{equation}
\frac{\partial\phi}{\partial t}=M\frac{\partial^{2}}{\partial x^{2}}\left(\frac{\delta\mathcal{H}}{\delta\phi}\right)+\sqrt{2MT}\frac{\partial}{\partial x}\Lambda(x,t),
\end{equation}
where $\Lambda(x,t)$ is Gaussian white noise with mean and variance:
\begin{align}
\left\langle \Lambda(x,t)\right\rangle  & =0\\
\left\langle \Lambda(x,t)\Lambda(x',t')\right\rangle  & =\delta(x-x')\delta(t-t').
\end{align}
In computer simulations, the space $x$ is discretized into lattice
with step size $\Delta x$:
\begin{equation}
x=i\Delta x\text{, where }i=0,1,2,\dots N_{x}-1,
\end{equation}
where $N_{x}$ is the total number of lattice sites. The system size
is then $L=N_{x}\Delta x$. Similarly, time is also discretized into:
\begin{equation}
t\rightarrow n\Delta t\text{, where }n=0,1,2,\dots,N_{t}-1,
\end{equation}
where $N_{t}$ is the total number of timesteps we are running the
simulation for. Consequently, the density field and the noise current
become:
\begin{align}
\phi(x,t) & \rightarrow\phi_{i}^{n}\\
\Lambda(x,t) & \rightarrow\Lambda_{i}^{n}
\end{align}
Next we need to regularize the Dirac delta function:
\begin{align}
\delta(x-x') & \rightarrow\frac{\delta_{i,i'}}{\Delta x}\\
\delta(t-t') & \rightarrow\frac{\delta_{n,n'}}{\Delta t}.
\end{align}
Thus we need to define a new noise:
\[
\tilde{\Lambda}_{i}^{n}=\sqrt{\Delta x\Delta t}\Lambda_{i}^{n},
\]
so that the correlation for this new noise is just a Kronecker delta:
\begin{equation}
\left\langle \tilde{\Lambda}_{i}^{n}\tilde{\Lambda}_{j}^{m}\right\rangle =\delta_{mn}\delta_{ij}.
\end{equation}

Recall the Hamiltonian functional:
\begin{equation}
\mathcal{H}[\phi]=\int_{0}^{L}dx\left\{ f(\phi)+\frac{\kappa}{2}\left(\frac{\partial\phi}{\partial x}\right)^{2}\right\} ,
\end{equation}
where $f(\phi)=\frac{a}{2}\phi^{2}+\frac{b}{4}\phi^{4}$. In discrete
space, the gradient operator becomes:
\begin{equation}
\frac{\partial\phi}{\partial x}\rightarrow\frac{\phi_{i+1}-\phi_{i-1}}{2\Delta x}+\mathcal{O}(\Delta x^{2}).
\end{equation}
Therefore, the Hamiltonian functional becomes:
\begin{align}
\mathcal{H}[\phi] & \rightarrow\mathcal{H}\{\phi_{i}\}\\
 & =\sum_{i=1}^{N_{x}-1}\Delta x\left\{ f(\phi_{i})+\frac{\kappa}{2}\left(\frac{\phi_{i+1}-\phi_{i-1}}{2\Delta x}\right)^{2}\right\} \\
 & =\sum_{i=1}^{N_{x}-1}\Delta x\left\{ f(\phi_{i})+\frac{\kappa}{8\Delta x^{2}}\left(\phi_{i+1}^{2}-2\phi_{i+1}\phi_{i-1}+\phi_{i-1}^{2}\right)\right\} 
\end{align}
The functional derivative in discrete space is defined to be (for
$d=1$):
\begin{align}
\frac{\delta\mathcal{H}}{\delta\phi} & \rightarrow\frac{1}{\Delta x}\frac{\partial\mathcal{H}}{\partial\phi_{i}}\\
 & =\frac{\partial}{\partial\phi_{i}}\sum_{j=1}^{N_{x}-1}\left\{ f(\phi_{j})+\frac{\kappa}{8\Delta x^{2}}\left(\phi_{j+1}^{2}-2\phi_{j+1}\phi_{j-1}+\phi_{j-1}^{2}\right)\right\} \\
 & =\sum_{j=1}^{N_{x}-1}\left\{ f'(\phi_{j})\delta_{ij}+\frac{\kappa}{8\Delta x^{2}}\left(2\phi_{j+1}\delta_{i,j+1}-2\phi_{j+1}\delta_{i,j-1}-2\phi_{j-1}\delta_{i,j+1}+2\phi_{j-1}\delta_{i,j-1}\right)\right\} \\
 & =f'(\phi_{i})+\frac{\kappa}{4\Delta x^{2}}(\phi_{i}-\phi_{i+2}-\phi_{i-2}+\phi_{i})\\
 & =f'(\phi_{i})-\kappa\left(\frac{\phi_{i+2}-2\phi_{i}+\phi_{i-2}}{4\Delta x^{2}}\right).
\end{align}
Now if we recall the continuum version of functional derivative,
\begin{equation}
\frac{\delta\mathcal{H}}{\delta\phi}=f'(\phi)-\kappa\frac{\partial^{2}\phi}{\partial x^{2}},
\end{equation}
the Laplacian operator should then be equal to:
\begin{equation}
\frac{\partial^{2}\phi}{\partial x^{2}}\rightarrow\frac{\phi_{i+2}-2\phi_{i}+\phi_{i-2}}{4\Delta x^{2}}+\mathcal{O}(\Delta x).
\end{equation}
Notice that the second derivative skips a lattice site, compared to
the first derivative.

Now putting everything together, the discretized dynamics has become:
\begin{align}
\phi_{i}^{n+1} & =\phi_{i}^{n}+\Delta t\,\frac{M}{4\Delta x^{3}}\left(\frac{\partial\mathcal{H}}{\partial\phi_{i+2}^{n}}-\frac{\partial\mathcal{H}}{\partial\phi_{i}^{n}}+\frac{\partial\mathcal{H}}{\partial\phi_{i-2}^{n}}\right)+\sqrt{\Delta t}\sqrt{\frac{MT}{2\Delta x^{3}}}(\tilde{\Lambda}_{i+1}^{n}-\tilde{\Lambda}_{i-1}^{n})
\end{align}
where $\{\tilde{\Lambda}_{i}^{n}\}$ are a set of independent Gaussian
random variables with zero mean and unit variance. Taking the limit
of continuous time, we can write the above equation as:
\begin{equation}
\frac{\partial\phi_{i}}{\partial t}=-\Gamma_{ij}\frac{\partial\mathcal{H}}{\partial\phi_{j}^{n}}+g_{ij}\tilde{\Lambda}_{j},
\end{equation}
where
\begin{align}
\Gamma_{ij} & =\frac{M}{4\Delta x^{3}}(2\delta_{i,j}-\delta_{i,j-2}-\delta_{i,j+2})\\
g_{ij} & =\sqrt{\frac{MT}{2\Delta x^{3}}}(\delta_{i,j-1}-\delta_{i,j+1}).
\end{align}
Now we can verify FDT
\begin{align}
g_{ik}g_{jk} & =\frac{MT}{2\Delta x^{3}}(\delta_{i,k-1}-\delta_{i,k+1})(\delta_{j,k-1}-\delta_{j,k+1})\\
 & =\frac{MT}{2\Delta x^{3}}(\delta_{i,j}+\delta_{i,j}-\delta_{i,j+2}-\delta_{i,j-2})\\
 & =2\Gamma_{ij}T,
\end{align}
or $gg^{T}=g^{T}g=2\Gamma T$.

For $d=2$ dimension, the spatial coordinates are:
\begin{align}
x & \rightarrow i\Delta x\text{, where }i=0,1,2,\dots,N_{x}-1\\
y & \rightarrow j\Delta y\text{, where }j=0,1,2,\dots,N_{y}-1,
\end{align}
The discretized Langevin equation is:
\begin{equation}
\phi_{ij}^{n+1}=\phi_{ij}+\Delta tM\nabla^{2}\mu_{ij}^{n}+\sqrt{\Delta t}\sqrt{\frac{2MT}{\Delta x\Delta y}}\nabla\cdot\boldsymbol{\Lambda}_{ij}^{n},
\end{equation}
where the gradient and Laplacian operator are:
\begin{align}
\frac{\partial\phi_{ij}}{\partial x} & =\frac{\phi_{i+1,j}-\phi_{i-1,j}}{2\Delta x}+\mathcal{O}(\Delta x^{2})\\
\frac{\partial\phi_{ij}}{\partial y} & =\frac{\phi_{i,j+1}-\phi_{i,j-1}}{2\Delta y}+\mathcal{O}(\Delta y^{2})\\
\nabla^{2}\phi_{ij} & =\frac{\phi_{i+2,j}-2\phi_{ij}+\phi_{i-2,j}}{4\Delta x^{2}}+\frac{\phi_{i,j+2}-2\phi_{ij}+\phi_{i,j-2}}{4\Delta y^{2}}+\mathcal{O}(\Delta x).
\end{align}
In Numpy, $\phi$ is represented as an array:
\begin{align*}
\phi & =\underset{}{}\left(\begin{array}{cccc}
\phi_{00} & \phi_{01} & \dots & \phi_{0,N_{y}-1}\\
\phi_{10} & \phi_{11} &  & \phi_{1,N_{y}-1}\\
\vdots &  & \ddots & \vdots\\
\phi_{N_{x}-1,0} & \phi_{N_{x}-1,1} & \dots & \phi_{N_{x}-1,N_{y}-1}
\end{array}\right)\downarrow x\text{-direction}\\
 & \quad\quad\quad\quad\quad\longrightarrow y\text{-direction}
\end{align*}

In simulation, it might be useful to track some macroscopic quantity
to check whether the simulation has reached a steady state or not.
For instance, we may measure the global average of the fluctuations
squared:
\[
A(t)=\frac{1}{V}\int_{V}\delta\phi(\mathbf{r},t)^{2}\,d\mathbf{r}
\]
In steady state, $\left\langle A\right\rangle _{s}$ Using the definition
of Fourier transform, we get:
\begin{align*}
A(t) & =\frac{1}{V}\int_{V}\sum_{\mathbf{q},\mathbf{q}'}\delta\phi_{\mathbf{q}}(t)\delta\phi_{\mathbf{q}'}(t)e^{i(\mathbf{q}+\mathbf{q}')\cdot\mathbf{r}}\,d\mathbf{r}\\
 & =\sum_{\mathbf{q}}|\delta\phi_{\mathbf{q}}|^{2}\\
 & \simeq\sum_{\mathbf{q}}S(q)
\end{align*}
Now we put the expression for the structure factor:
\begin{align*}
A(t) & =\sum_{\mathbf{q}}\frac{T}{a+3b\phi_{0}^{2}+\kappa q^{2}}\\
 & \simeq\frac{V}{(2\pi)^{d}}\int_{0}^{q_{\text{max}}}dq\,\Omega_{d}q^{d-1}\frac{T}{a+3b\phi_{0}^{2}+\kappa q^{2}}
\end{align*}
In particular, for $d=2$ case,
\[
\]


\section{Using fast Fourier transform in Numpy}

The method to perform a $2$-dimensional discrete Fourier transform
on the array phi and save it to another array phi\_q is:
\begin{lstlisting}
phi_q = numpy.fft.fft2(phi, norm='ortho')
\end{lstlisting}
The discrete Fourier transform in Numpy is defined to be:
\begin{align}
\phi(\mathbf{r}) & =\frac{1}{\sqrt{N_{x}N_{y}}}\sum_{\mathbf{q}}\phi_{\mathbf{q}}e^{i\mathbf{q}\cdot\mathbf{r}}\quad\text{(inverse Fourier transform)}\\
\phi_{\mathbf{q}} & =\frac{1}{\sqrt{N_{x}N_{y}}}\sum_{\mathbf{r}}\phi(\mathbf{r})e^{-i\mathbf{q}\cdot\mathbf{r}}\quad\text{(forward Fourier transform)}.
\end{align}
But we want:
\begin{align}
\phi(\mathbf{r}) & =\frac{1}{\sqrt{L_{x}L_{y}}}\sum_{\mathbf{q}}\phi_{\mathbf{q}}e^{i\mathbf{q}\cdot\mathbf{r}}\\
\phi_{\mathbf{q}} & =\frac{1}{\sqrt{L_{x}L_{y}}}\underbrace{\sum_{\mathbf{r}}\Delta x\Delta y}_{\int d\mathbf{r}}\,\phi(\mathbf{r})e^{-i\mathbf{q}\cdot\mathbf{r}}.
\end{align}
so we need to multiply the forward Fourier transform in Numpy by $\sqrt{\Delta x\Delta y}$
and divide the inverse Fourier transform in Numpy by $\sqrt{\Delta x\Delta y}$.
Also note that the array $\phi_{q}$ is arranged in a peculiar way
in Numpy:
\[
\phi_{q}=\underbrace{\begin{array}{|c|c|c|c|c|c|c|c|c|}
\hline \phi_{0} & \phi_{\frac{2\pi}{L}} & \phi_{\frac{2\pi(2)}{L}} & \dots & \phi_{\frac{2\pi(N/2-1)}{L}} & \phi_{\frac{2\pi(-N/2)}{L}} & \phi_{\frac{2\pi(-N/2+1)}{L}} & \dots & \phi_{\frac{2\pi(-1)}{L}}\\\hline \end{array}}_{\text{total length}=N}
\]
So we also need to roll the elements of the array to the right by
$N/2$.

\section{Pseudo-spectral simulation}

The equation we are solving is:
\begin{equation}
\frac{\partial\phi}{\partial t}=Ma\nabla^{2}\phi+Mb\nabla^{2}\phi^{3}-M\kappa\nabla^{4}\phi+\sqrt{2MT}\nabla\cdot\boldsymbol{\Lambda}
\end{equation}
Taking Fourier transform, we get:
\begin{equation}
\frac{\partial\phi_{\mathbf{q}}}{\partial t}=-\left(Maq^{2}+M\kappa q^{4}\right)\phi_{\mathbf{q}}-Mbq^{2}\mathcal{F}[\phi^{3}]_{\mathbf{q}}+\sqrt{2MT}i\mathbf{q}\cdot\boldsymbol{\Lambda}_{\mathbf{q}}.
\end{equation}
The algorithm will be:
\[
\begin{array}{c|c|c}
\text{Real space} & \longleftrightarrow & \text{Fourier space}\\
\hline \text{calculate }\phi\text{, }\phi^{3}\text{ and }\boldsymbol{\Lambda} & \underset{\text{forward FFT}}{\longrightarrow} & \phi_{\mathbf{q}}\text{, }\mathcal{F}[\phi^{3}]_{\mathbf{q}}\text{ and }\boldsymbol{\Lambda}_{\mathbf{q}}\\
 &  & \text{update }\downarrow\text{ timestep}\\
\text{get }\phi\text{ at the next timestep} & \underset{\text{inverse FFT}}{\longleftarrow} & \phi_{\mathbf{q}}\text{ at the next timestep}
\end{array}
\]

\end{document}
